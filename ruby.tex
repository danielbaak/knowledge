\documentclass{article}

\begin{document}
\author{Florian Kaufmann}
\title{Ruby Summary}
\date{\today}
\maketitle
\tableofcontents

\section{Introduction}

\section{Structure of a Ruby program}

\begin{verbatim}
## RUBY
## ---------------------------------------------------------------------------
        
RubyScript = Body;       

## 1) Not exact because none of the LoopJmpExp are allowed
Body = %simple#^1 BodyLoopJmp;

## 1) Not exact because a \n doesn't terminate a statement if it's obviusly
## incomplete, e.g. has a trailing binary operator.
BodyLoopJmp = %simple#%1 ( (Exp|SpecialBlock) (\n|\;){l?} )*;     

## 1) Not exact because of the LoopJmpExp only retry is allowed
BodyRetry = %simple#^1 BodyLoopJmp;     
            
SpecialBlock = ('BEGIN | 'END) `{ Body `} ;
              
## Expressions / Operators
## ---------------------------------------------------------------------------
              
Exp = DefExp | FlowCtrlExp | MethCallExp | OpExp | Term;

OpExp = Exp BinOp Exp | UnPreOp Exp | Exp UnPostOp | TernaryOp;
  
## Similar to "if Bool then Exp else Exp end"
TernaryOp = Bool `? Exp#IfTrue `: Exp#IfFalse ;
  
Term = `( Exp `) | GenDelInput | TermLit | TermMisc | ExtName;
              
BinOp = `; | ...;
              
%ruleprecedence =               
              
## 
## ---------------------------------------------------------------------------
Integer= Exp; ## responds to to_int
String = Exp; ## responds to to_str
Array  = Exp; ## responds to to_ary
Hash   = Exp; ## responds to to_hash
Symbol = Exp; ## responds to to_sym
       
Bool   = Exp; ## Exp is in boolean context       
       
Class  = Exp; ## of class Class
       
Regex  = Exp; ## of class Regex
       
## Flow control / error handling
## ---------------------------------------------------------------------------
## See also
## - the ternary operator ?:   
## - In ruby doc: Kernel.(catch|throw|raise|fail|exit|abort)
        
FlowCtrlExp = IfExp | UnlessExp | CaseExp | LoopExp | ModifiedExp |
  BeginEndBlock | LoopJmpExp;   

IfExp =
  'if Bool Then Body 
  ('elsif Bool Then Body)*
  ['else Body ]
  'end;
  
UnlessExp =  
  'unless Bool Then Body
  ['else Body]
  'end;

## Each comparision is tested against target with comparsion===target
CaseExp =
  'case Exp#Target
  ('when Exp#Comparison Then Body)+
  ['else Body]
  'end;

LoopExp =
  ('while|'until) Bool Do BodyLoopJmp 'end |
  
  ## infinite loop. loop is actually an iterator
  'loop CodeBlock0Arg#Body |
  
  ## Actually only syntax sugar for "(Exp).each do |CBArgs| Body end"
  'for CBArgs 'in Exp#Collection Do BodyLoopJmp 'end;
  
ModifiedExp =
  ## while/until: If the body is an BeginEndBlock, the test expression is only
  ## evaluated after the body has been executed for the first time. Else, the
  ## test expression is evaluated before a possible execution of the body.
  Exp#Body ('if|'unless|'while|'until) Bool |
  
  ## Rescued are StandartError and its children
  Exp 'rescue Exp;
            
## Not everywhere these keywords are allowed. See Body,BodyRetry,BodyLoopJmp
LoopJmpExp =
  'break [Arg]| ## continue at stmt following loop
  'redo       | ## repeat loop from start, without condition/inc
  'retry      | ## repeat loop from start, again evaluation condition
  'next [Arg] ; ## skip to end of loop, i.e. start next iteration
  
  Arg = @\( Exp @\)?; ## Return value of block
            
BeginEndBlock = 'begin BodyWithErrorHandling 'end;
        
## See Kernel.raise how to raise exceptions
## 1) When an excpetion is raised in code, the next 'enclosing' rescue is
## searched, until one is found for which's parameter "parameter===$!" is true.
## 2) If a Name is given after the param list, it is set to $!.
BodyWithErrorHandling =
  Body#Code
  ['rescue#^1 RescueParamList [`=> Name]#^2 Then BodyRetry ]
  ['else Body]    ## if no exception raised in code
  ['ensure Body]; ## always
            
  RescueParamList = ( Class `, {l0} )*;
  
Then =  'then | `: | \n;

Do = 'do | \: | \n;
  
## Literals
## ---------------------------------------------------------------------------
   
%sidetree              
Lit =
  TermLit |
  RangeLit | ## created using operators .. / ... 
  RegexLit ; 

TermLit = NumLit | StringLit | ArrayLit | HashLit | SymbolLit | BuiltinLit; 
                       
NumLit = IntLit | FloatLit | CharIntVal;

IntLit %=
  '[1-9][\_?(DecDigit\_{l0})+] | ## decimal based 
  "0b"      (BinDigit\_{l0})*  | ## binary based 
  "0"       (OctDigit\_{l0})*  | ## octal based 
  "0x"      (HexDigit\_{l0})*  ; ## hex based 

FloatLit %=
  %simple
  (DecDigitSeq |. \.DecDigitSeq) ["e"DecDigitSeq];
              
## The integer value of the given char
CharIntVal %= \?(\\'[CM]\-)*\A;
              
## see also w/W quote, which are also sort of array literals
ArrayLit = `[ ( exp `, {l?} )* `] ;
             
## The key obj must respond to the hash method
HashLit = `{ Val `} | "Hash[" Val `] ;
  Val = ( exp#key ( `, | `=> ) exp#value `, {l?} )*;        
              
## The start & end obj must be comparable with their <=> operator and must
## support the succ method.             
RangeLit = exp #start (`.. #EndExcl | `... #EndIncl ) exp #end; 

## see also q/Q quote, which are also sort of string literals
StringLit = SimpleStringLit | HereString |
  SimpleStringLit SimpleStringLit#*implicit concatenation*#;
              
## See also GenDelInput
SimpleStringLit %=              
  %simple              
  \' ... \'  | ## single quote
  \" ... \"  ; ## double quote
  
## The following construct will be replaced by the here document. It the - is
## present, the delimiter might be indented. If the delimiter is within quotes,
## the here document is altered according to the respective quoting rules.
HereString %= `<< `- ? (Identifier|SimpleStringLit)#del; 
  
SymbolLit %= `: Name; 
  
## see also the %r GenDelInput
RegexLit %= `/ ... `/ Option*;   
  Option = 'i #CaseInsensitive | 'o #SubstituteOnce | 'm #Multiline |
    'x #ExtendedMode;
  
BuiltinLit =                       
  'nil   | ## singleton instance of class NilClass
  'false | ## singleton instance of class FalseClass
  'true  | ## singleton instance of class TrueClass
  '__FILE__ | ## name of current source file
  '__LINE__;  ## the current line number in the source file
  
## General delimited input.
## 1) Can spawn over newlines.
GenDelInput =
  %simple
  \%(QuoteType|>'Q)QuoteDel ... #^1 ~~QuoteDel 
  %exact              
  ## ??? defaullt Q
  %tok("%q" QuoteDel qBody ~~QuoteDel) |
  %tok("%Q" QuoteDel QBody ~~QuoteDel) |
  %tok("%w" QuoteDel wBody ~~QuoteDel) |
  %tok("%W" QuoteDel WBody ~~QuoteDel) |
  %tok("%r" QuoteDel rBody ~~QuoteDel) |
  %tok("%x" QuoteDel xBody ~~QuoteDel) ;
              
## If ( { { < the closing delimiter is the matching delimiter, taking into
## account nested delimiter pairs. Even spaces, newlines are allowed as
## delimiter.
QuoteDel %= \A;
              
QuoteType %=
  'q | ## as string literal with single quotes
  'Q | ## as string literal with double quotes
  'w | ## Space separated word list -> array of strings. Space might be escaped with \.
       ## Literal backslashes are \\
  'W | ## Additionally, all the normal double-quoted string substitutions are
       ## made.
  'r | ## Regex Pattern
  'x;  ## Shell Command 

## Classes / Methods
## ---------------------------------------------------------------------------

DefExp = ClassDef | MethodDef;   
   
ClassDef =
  'class ConstName (\N|\;)
  ClassBody
  'end;   
   
## Not exact because body may not contain class/module definitions. 
MethodDef =
  'def DefName MethodArgs (\N|\;)
  %simple#^1 BodyWithErrorHandling %end
  'end;
          
  ## convention is to use parantheses when there are parameters and leave them
  ## away when there aren't any parameters 
  MethodArgs = @\( Args |.<\,> ArgsWD |.<\,> Rest |.<\,> Proc @\)?;
             
  Args  = ( LocalName \,{l0} )+;
  ArgsWD= ( LocalName `= Exp#Init \,{l0} )+;
  Rest  = \*LocalName;
  Proc  = \&LocalName;
  
## If Exp is given, it's a singleton method of the obj given by Exp's value.
DefName = [Exp `. ]MethodName; 
  
MethodName = SimpleMethodName | OpName;

OpName = `[] | `[]= | `** | `<< | `>> | `<= | `=> | `<=> | `== | `=== | `!= |
  `=~ | `!~ | `+@ | `-@ | '[!~+\-*/%&\^|<>];
  
MethCallExp = ;  
  
CodeBlock = `{ CBArgsOuter Body `} | 'do CBArgsOuter BodyLoopJmp 'end; 
 CBArgsOuter = [ `| CBArgs `| ];
 
CBArgs = Args |.<\,> Rest;
 Args = ( Name `, {l0} )+;
 Rest = \*Name;
              
## Number of arguments the block will receive
CodeBlock0Arg = Block;
CodeBlock1Arg = Block;
CodeBlock2Arg = Block;
              
## ???
## ---------------------------------------------------------------------------
ExtName =
  Name  |
  'self |
  'super;
              
## Lexxer
## ---------------------------------------------------------------------------
        
%patternprecedence = Keywords, Name, 
        
Keywords %= '__FILE__ | '__LINE__ | 'BEGIN | 'END | 'alias | 'and | 'begin |
  'break | 'case | 'class | 'def | "defined?" | 'do | 'else | 'elsif | 'end |
  'ensure | 'false | 'for | 'if | 'in | 'module | 'next | 'nil | 'not | 'or |
  'redo | 'rescue | 'retry | 'return | 'self | 'super | 'then | 'true | 'undef |
  'unless | 'until | 'when | 'while | 'yield;
              
Name %=
  l n*        | ## local variable | wordsep _
  N '[?!-]?   | ## simple method name | should start with lowercase
  `@ N        | ## instance variable | start with l
  `@@ N       | ## class variable | 
  u n*        | ## constant, class, module. Constant use _ , others CamelCase.
  `$ N        | ## global variable
  `$ d        | ## standart global variable
  `$ '[^\w\s] | ## dito
  `$- n       ; ## dito
 
  l %= '[a-z_];  ## lower case letter
  u %= '[A-Z];   ## upper case letter
  d %= '[0-9];   ## digit
  n %= l|u|d;    ## name character
  N %= (l|u) n*; ## name
         
DecDigit %= '[0-9];
BinDigit %= '[01];
OctDigit %= '[0-7];
HexDigit %= '[0-9a-fA-F];
         
%discard         
Comment %= `# \A*? $$;

%discard         
Docu %= $^ "=begin" \A*? $^ "=end" \A*? $$

%discard         
DataSection %= $^"__END__"$$ \A*? $';

              
            
#* Todo:
- ClassBody
- catch (a Kernel method)
- lambda
- Kernel
- Iterators
- `` backquote (a Kernel method)
- alias
- yield
- defined?
- undef  
- module *#
\end{verbatim}

\end{document}
