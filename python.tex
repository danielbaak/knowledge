\documentclass{article}

\begin{document}
\author{Florian Kaufmann}
\title{Python Summary}
\date{\today}
\maketitle
\tableofcontents

\section{Introduction}

\section{Yasmala}

\begin{verbatim}
## literals
## ----------------------------------------------------------------------
        
Lit = NumLit | StringLit | ByteLit;        
        
NumLit = IntLit | FloatLit | ImagLit;        
        
IntLit %=
 '0+               | 
 '[1-9]  Digit*    | 
 '0'[bB] BinDigit+ | 
 '0'[oO] OctDigit+ | 
 '0'[xX] HexDigit+ ;
 
FloatLit %= PointFloat | ExpFloat;
  PointFloat %= Digit* \. Digit+ | Digit+ \.;
  ExpFloat %= (Digit+|PointFloat) '[eE]'[+-]?Digit+;
        
ImagLit %= (FloatLit|Digit+)'[jJ];

## Adjancted string literals are concatenated
StringLit = SimpleStringLit+;
        
## There are 3 binary 'switches'
## 1) Single quote or double quote doesn't really make a difference.
## 2.1) no prefix: escape seq allowd
## 3.2) raw prefix: no esscape seq. 
## 4) triple-quoted strings: Unqouted newlines are allowed.
SimpleStringLit %=
  '[rR] @( `"   | `'   )  ( RawStrEsc | \a )*?  ~~@ |
  '[rR] @( `""" | `''' )  ( RawStrEsc | \A )*?  ~~@ |
        @( `"   | `'   )  ( StrEscSeq | \a )*?  ~~@ |
        @( `""" | `''' )  ( StrEscSeq | \A )*?  ~~@ ;
  
## the backslash and the following char are both literaly part of the literal
RawEsc %= \\\a;
       
EscSeq %=
  "\\\n" | ## discards both the backslash and the newline
  "\\\\" | ## Backslash 
  "\\'"  | ## Single quote 
  "\\""  | ## Double quote 
  "\\a"  | ## ASCII Bell (BEL)                  
  "\\b"  | ## ASCII Backspace (BS)              
  "\\f"  | ## ASCII Formfeed (FF)               
  "\\n"  | ## ASCII Linefeed (LF)               
  "\\r"  | ## ASCII Carriage Return (CR)        
  "\\t"  | ## ASCII Horizontal Tab (TAB)        
  "\\v"  | ## ASCII Vertical Tab (VT)           
  "\\"OctDigit{1,3} | ## Character with octal value
  "\\x"HexDigit{1,2}; ## Character with hex value 
   
StrEscSeq %=
   EscSeq |
   "\\N{" \a+?#name "}" | ## character name in the unicode database 
   "\\u" HexDigit{4}    | ## charcter with 16-bit hex value
   "\\U" HexDigit{8}    ; ## charcter with 32-bit hex value
   
## statements
## ----------------------------------------------------------------------
Stmt =  (SimpleStmt `; {l?} )+ NL | CompoundStmt; 
     
CompoundStmt = CondStmt | LoopStmt | DefStmt | TryStmt | WithStmt;
             
SimpleStmt = expr_stmt | del_stmt | pass_stmt | flow_stmt |
             import_stmt | global_stmt | nonlocal_stmt | assert_stmt;             
             
DefStmt = FuncDef | ClassDef;         
             
Suite =
  (SimpleStmt `; {l?} )+ NL |
  NL Indent Stmt+ Dedent;
        
## control flow
## ----------------------------------------------------------------------
CondStmt = IfStmt;
LoopStmt = WhileStmt | ForStmt;
        
IfStmt =
  'if Bool `: Suite
  ('elif Bool `: Suite )*
  ['else `: Suite ];
        
WhileStmt =
  'while Bool `: Suite
  ['else `: Suite ] 
  
ForStmt =
  'for TargetList 'in Iterable `: Suite
  ['else `: Suite ];
        
TryStmt = TryStmt1 | TryStmt2;
        
TryStmt1 =
  'try `: Suite
  ('except [Exp ['as Taget]] `: Suite)+
  ['else `: Suite]
  ['finally `: Suite];
  
TryStmt2 =
  'try `: Suite;

WithStmt = 'with Exp 'as Target `: Suite;

## definitions
## ----------------------------------------------------------------------
funcdef = 
  Decorator*        
  'def Name `( Params `) [`-> Test] `: Suite;
  
  Params  = (defparameter `, )*
                 = (  `* [parameter] ("," defparameter)*
                 = [, `** parameter]
                 = | `** parameter
                 = | defparameter [`, ] );
  parameter      = identifier [`: expression];
  defparameter   = parameter [`= expression];
   
Decorator = `@ DName [ `( [Args] `) ] NL;
   
DName = (Name `. {l0})+;
      
Args = Arg `,    
   
## lexxer        
        
%discard        
Comment %= `# \a* \N;
        
## If more blanks than current indent level, else discard
Indent %= %simple $$ '[ \t]* ;
       
## If less blanks than current indent level, else discard
Dedent %= %simple $$ '[ \t]* ;
       
%discad        
Blanks %= '[ \t\f]+ | BlankLine;
       
%discard       
EscapedNewline %= \\\n;       
               
BlankLine %= $^ \s* $$
       
NL %= \n;

        
Keywords %= 'False | 'class | 'finally | 'is | 'return | 'None |
 'continue |'for | 'lambda | 'try | 'True | 'def | 'from |
 'nonlocal | 'while | 'and | 'del | 'global | 'not | 'with |
 'as | 'elif | 'if | 'or | 'yield | 'assert | 'else | 'import |
 'pass | 'break | 'except | 'in | 'raise;
 
ReservedName %=
 '_\w+ |     ## ???
 '__\w+'__ | ## system-defined names
 '__\w+;     ## class private names
 
 
## Note: Python 3.0 introduces additional chars from outside the ASCII range.
Name %= (\w-\d)\w+; ## length is really unlimted
        
 %lexerorder: Indent, Dedent, Blanks, Keywords, ReservedName, Name\end{verbatim}


\end{document}
