%%%%%%%%%%%%%%%%%%%%%%%%%%%%%%%%%%%%%%%%%%%%%%%%%%%%%%%%%%%%%%%%%%%%%%%%%%%%%%%%
% Title & Abstract
%%%%%%%%%%%%%%%%%%%%%%%%%%%%%%%%%%%%%%%%%%%%%%%%%%%%%%%%%%%%%%%%%%%%%%%%%%%%%%%%

\documentclass[a4paper]{report}
\usepackage{verbatim}
\begin{document}

\author{Florian Kaufmann}
\title{Subversion Summary}
\date{\today}
\maketitle


%%%%%%%%%%%%%%%%%%%%%%%%%%%%%%%%%%%%%%%%%%%%%%%%%%%%%%%%%%%%%%%%%%%%%%%%%%%%%%%%
\tableofcontents
%%%%%%%%%%%%%%%%%%%%%%%%%%%%%%%%%%%%%%%%%%%%%%%%%%%%%%%%%%%%%%%%%%%%%%%%%%%%%%%%

%%%%%%%%%%%%%%%%%%%%%%%%%%%%%%%%%%%%%%%%%%%%%%%%%%%%%%%%%%%%%%%%%%%%%%%%%%%%%%%%
\chapter{Introduction}
%%%%%%%%%%%%%%%%%%%%%%%%%%%%%%%%%%%%%%%%%%%%%%%%%%%%%%%%%%%%%%%%%%%%%%%%%%%%%%%%

Short overview of tools that come with subversion.

\begin{description}
 \item[svn] The command-line client program.           
 \item[svnversion] A program for reporting the state (in terms of revisions
 of the items present) of a working copy.
 \item[svnlook] A tool for directly inspecting a Subversion repository.  
 \item[svnadmin] A tool for creating, tweaking or repairing a Subversion
 repository.
 \item[svndumpfilter] A program for filtering Subversion repository dump
 streams.
 \item[mod_dav_svn] A plug-in module for the Apache HTTP Server, used to make
 your repository available to others over a network.
 \item[svnserve] A custom standalone server program, runnable as a daemon
 process or invokable by SSH. Another way to make your repository
 available to others over a network.
 \item[svnsync] A program for incrementally mirroring one repository to
 another over a network.
\end{description}

%%%%%%%%%%%%%%%%%%%%%%%%%%%%%%%%%%%%%%%%%%%%%%%%%%%%%%%%%%%%%%%%%%%%%%%%%%%%%%%%
\chapter{Fundamental Concepts}
%%%%%%%%%%%%%%%%%%%%%%%%%%%%%%%%%%%%%%%%%%%%%%%%%%%%%%%%%%%%%%%%%%%%%%%%%%%%%%%%

SVN's core is a \textbf{repository}, a central store of data.  It's similar
to a file server, but it rembers every change made.  

A \textbf{conflict} is a situation where two clients check in the same file
they have checked out earlier, and their changes do overlap.  Then the
the later checking in has to \textbf{resolve} the conflict.  This is only
possible with line based text files.  In case of binary files, they have
to be \textbf{locked}, that is the file can only be checked out by the one
client that locked it.  

A \textbf{working copy} is an ordinary directory tree on your local system
where you can work with the files as you like.  Subversion does only do
something with these files if you explicitely tell it to do so.  The
working copy usually corresponds to particular subtree/subdirectory of
the repository.  

To get a working copy, you must \textbf{check out} some subtree of the
repository. When you finished making your changes, you \textbf{commit} (also
called \textbf{checking in} or \textbf{publishing} or \textbf{making public}) to the
repository.  \textbf{Updating} means to incorporate the latest version in
the repository into the working directory.

Transactions/changes to/from the repository are \textbf{atomic}, i.e. either
all or none of the commanded changes happen. Eeach succesfull commit to
the repository creates a new \textbf{revision}. Eeach revision is a natural
number, incrementing by 1 with each successfull commit, and applies to a
whole subtree in the repository, i.e. not to an individual file.

A working copy contains some extra files, created and maintained by
Subversion, in the \textbf{administrative directory} ``.svn''. 

A file is \textbf{locally (un)changed}, if changes have been made to the
copy in the working directory. Its \textbf{current}, if no changes to
that file have been committed to the repository since its working
revision. It's \textbf{out-of-date} otherwise.

%%%%%%%%%%%%%%%%%%%%%%%%%%%%%%%%%%%%%%%%%%%%%%%%%%%%%%%%%%%%%%%%%%%%%%%%%%%%%%%%
\chapter{Example}
%%%%%%%%%%%%%%%%%%%%%%%%%%%%%%%%%%%%%%%%%%%%%%%%%%%%%%%%%%%%%%%%%%%%%%%%%%%%%%%%

Create a new repository and 'initialize' it with local files you already have.

\begin{verbatim}
$ svnadmin create /usr/local/svn/myproject
$ svn import ~/src/myproject file:///usr/local/svn/myproject
\end{verbatim}

A client which doesn't have yet anything can now start by checking out
the project's files from the repository to get his local working copy of
the files.

\begin{verbatim}
$ svn checkout file:///usr/local/svn/myproject ~/src
\end{verbatim}

He can now do arbitrary changes to these files. 

The client can also make changes to the 'file system' using 
\verb@svn (add | delete | copy | move | mkdir)@. 


\begin{description}
 \item[A] The item has been scheduled for \textbf{addition} into the
            repository.
 \item[D] The item has been scheeduled for \textbf{deletion}
 \item[C] The item is in state of \textbf{confict}, i.e. changes received
            from the repository overlap with local changes.
 \item[M] The item has been \textbf{modified}.
\end{description}


\end{document}
